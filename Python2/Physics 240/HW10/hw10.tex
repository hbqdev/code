% Homework 10.tex 

\documentclass{article}
\usepackage{graphicx} % for figures
\usepackage{float}
\usepackage[export]{adjustbox}
\usepackage{fancyhdr}
\begin{document}

\title{Homework 9 - Physics 240\\
		Gauss and Kirchhoff}
\author{Tin Tran}

\maketitle

\section{Introduction}
The purpose of this excerise to practice Gaussian elimination method to solve a system of linear of equation, in this case, the system of equations is constructed by applying Kirchohff's laws of current to the circut given in the homework. The system equation I come up with is :
\begin{center}
E$_1$ - I$_1$R$_1$ - I$_2$R$_3$ - E$_2$ - I$_1$R$_2$ = 0\\
-I$_3$R$_4$ - E$_3$ - I$_3$R$_5$ + E$_2$ + I$_2$R$_3$ = 0\\
I$_1$ - I$_2$ - I$_3$ = 0
\end{center}
Applying the values for R$_1$ = R$_2$ = 1, R$_3$ = R$_4$ = 2, R$_5$ = 5, E$_1$ = 2, E$_3$ = 5, and E$_2$ = 20 I get the following equations
\begin{center}
I$_1$ + I$_2$ + 0 = -9\\
7I$_3$ - 2I$_2$ = 15\\
I$_1$ - I$_2$ - I$_3$ = 0
\end{center}
Putting the matrix form of this system of equation into my program and I get
\begin{center}
I$_1$ = -4.125\\
I$_2$ = -4.875\\
I$_3$ = 0.75\\
\end{center}
The answers are verified with numpy.linalg, direct substitution, and reduced row echelon form methods, all of which gave the same answers.
\end{document}